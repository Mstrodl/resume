%%%%%%%%%%%%%%%%%
% This is an sample CV template created using altacv.cls
% (v1.1.2, 1 February 2017) written by LianTze Lim (liantze@gmail.com). Now compiles with pdfLaTeX, XeLaTeX and LuaLaTeX.
%
%% It may be distributed and/or modified under the
%% conditions of the LaTeX Project Public License, either version 1.3
%% of this license or (at your option) any later version.
%% The latest version of this license is in
%%    http://www.latex-project.org/lppl.txt
%% and version 1.3 or later is part of all distributions of LaTeX
%% version 2003/12/01 or later.
%%%%%%%%%%%%%%%%

%% If you need to pass whatever options to xcolor
\PassOptionsToPackage{dvipsnames}{xcolor}

%% If you are using \orcid or academicons
%% icons, make sure you have the academicons
%% option here, and compile with XeLaTeX
%% or LuaLaTeX.
% \documentclass[10pt,a4paper,academicons]{altacv}

%% Use the "normalphoto" option if you want a normal photo instead of cropped to a circle
% \documentclass[10pt,a4paper,normalphoto]{altacv}

\documentclass[10pt,letter]{altacv}

%% AltaCV uses the fontawesome and academicon fonts
%% and packages.
%% See texdoc.net/pkg/fontawecome and http://texdoc.net/pkg/academicons for full list of symbols.
%%
%% Compile with LuaLaTeX for best results. If you
%% want to use XeLaTeX, you may need to install
%% Academicons.ttf in your operating system's font
%% folder.


% Change the page layout if you need to
\geometry{left=1cm,right=9cm,marginparwidth=7.25cm,marginparsep=0.75cm,top=0.5cm,bottom=1cm}

% Change the font if you want to.

% If using pdflatex:
\usepackage[utf8]{inputenc}
\usepackage[T1]{fontenc}
\usepackage[default]{lato}
\usepackage[none]{hyphenat}
\usepackage[document]{ragged2e}

% If using xelatex or lualatex:
% \setmainfont{Lato}

% Change the colours if you want to
\definecolor{AccentGreen}{HTML}{339966}
\definecolor{AcePurple}{HTML}{800080}
\definecolor{DarkerPurple}{HTML}{5b005b}
\definecolor{SlateGrey}{HTML}{2E2E2E}
\definecolor{LightGrey}{HTML}{666666}
\definecolor{SomeBlue}{HTML}{4286f4}
\colorlet{heading}{AcePurple}%Sepia}
\colorlet{accent}{DarkerPurple}%AccentGreen}
\colorlet{emphasis}{SlateGrey}
\colorlet{body}{LightGrey}

% Change the bullets for itemize and rating marker
% for \cvskill if you want to
\renewcommand{\itemmarker}{{\small\textbullet}}
\renewcommand{\ratingmarker}{\faCircle}

\begin{document}
\name{Galen Guyer}
\tagline{Software Engineer Seeking a Spring or Summer 2022 Co-op}
\personalinfo{%
  % Not all of these are required!
  % You can add your own with \printinfo{symbol}{detail}
  \email{galen@galenguyer.com}\\\smallskip
  \phone{(425) 898-3145}\\\smallskip
  \homepage{galenguyer.com}\\\smallskip
  \github{github.com/galenguyer}\\\smallskip
  \linkedin{in/galenguyer}\\\smallskip
  %% You MUST add the academicons option to \documentclass, then compile with LuaLaTeX or XeLaTeX, if you want to use \orcid or other academicons commands.
%   \orcid{orcid.org/0000-0000-0000-0000}
}

%% Make the header extend all the way to the right, if you want.
\begin{fullwidth}
\marginpar{\makesidebarheader\medskip

\cvsection{Education}
\cvevent{Rochester Institute of Technology}{Software Engineering B.S.}{Expected Graduation May 2024}{}

\medskip

\cvsection{Skills}
\cvsubsection{Languages}
\cvtag{C\#}
\cvtag{Python}
\cvtag{HTML/CSS}
\cvtag{JavaScript}
\cvtag{Go}
\cvtag{React}
\cvtag{PowerShell}
\cvtag{Rust}
\cvtag{Java}
\cvtag{PHP}

\smallskip
\smallskip

\cvsubsection{Tools}
\cvtag{NGINX}
\cvtag{Linux}
\cvtag{Git}
\cvtag{GitHub}
\cvtag{Docker}
\cvtag{Docker Compose}
\cvtag{Traefik}
\cvtag{Azure}
\cvtag{Bash}
\cvtag{MongoDB}
\cvtag{Flask}
\cvtag{Ansible}
\cvtag{Terraform}
\cvtag{Kubernetes}
\cvtag{AWS Infinidash}

%% Yeah I didn't spend too much time making all the
%% spacing consistent... sorry. Use \smallskip, \medskip,
%% \bigskip, \vpsace etc to make ajustments.
\medskip
\cvsection{Activities}

\cvevent{Computer Science House}{Systems Administrator}{April 2020 - Present}{}
Maintain House Services such as the student-run server room and all services within it, including virtual machine and user profile management portals

\divider

\cvevent{WITR Radio}{Internal Developer}{October 2019 - Present}{}
Responsible for the upkeep and development of internal and public facing services such as the public website
%%\cvevent{Accessibility Learning Labs}{Software Engineer}{May 2020 - August 2020}{}
%%Worked in a federally funded research program to develop React apps to demonstrate the importance of accessible software to students and developers

\divider

\cvevent{Society of Software Engineers}{Public Relations}{January 2020 - May 2020, January 2021 - Current}{}
Coordinated company visits and tech talks throughout the semester

\divider

\cvevent{Boy Scouts}{Eagle Scout}{September 2017}{}
Achieved Boy Scouts' highest rank after earning 26 merit badges and designing, coordinating, and leading a 100+ hour service project

}
    \vspace*{-1\baselineskip}
\makecvheader
\end{fullwidth}
%% Provide the file name containing the sidebar contents as an optional parameter to \cvsection.
%% You can always just use \marginpar{...} if you do
%% not need to align the top of the contents to any
%% \cvsection title in the "main" bar.

\cvsection{Experience}

\cvevent{Blackbaud}{DevOps Intern}{May 2021 -- August 2021}{Remote}
\begin{itemize}
  \item Worked with the Splunk API to automate creation and updating of alerts from version-controlled files
  \item Designed sane defaults to be loaded by a script to ease creation and updating of alerts in bulk
\end{itemize}
\textit{\textbf{Tools:} PowerShell, Splunk}

\divider

\cvevent{Microsoft}{Software Engineering Intern - One Customer Voice Team}{June 2019 -- August 2019}{Redmond, WA}
\begin{itemize}
\item Improved item grouping for the internal feedback aggregation tool
\item Exposed all previously hidden top level fields and automatically detected field type via ElasticSearch mappings, greatly increasing both the granularity and flexibility for users
\item Implemented frontend, backend, and tests with 100\% backend code coverage
\end{itemize}
\textit{\textbf{Tools:} C\#, ASP.NET, Azure Service Fabric, ElasticSearch, AngularJS}

\cvsubevent{Software Engineering Intern - Office Security Penetration Testing Team}{June 2018 -- August 2018}{Redmond, WA}
\begin{itemize}
\item Developed a PowerShell script to create a Windows VM, automatically install Office, gather debugging symbols for Office, and send these to a Microsoft \\ Security Risk Detection server to install and start a fuzzing run
\item Implemented a server to collect, deduplicate, and report new bugs
\item Worked closely with the Microsoft Security Risk Detection team to help improve the job deployment process while the tool was in internal beta
\end{itemize}
\textit{\textbf{Tools:} PowerShell, Microsoft Security Risk Detection, Azure, ASP.NET}

\smallskip 

\cvsection{Projects}

%\project{GenericBot}{https://github.com/galenguyer/GenericBot}
%\begin{itemize}
%\item Built a Discord bot to provide moderation tools and fun commands to over 50 servers totaling over 16,000 users
%\item Used C\# to connect to Discord's API and store all user data in an encrypted, self-hosted MongoDB instance
%\item Exposed user-stored quotes via a webpage ASP.NET Razor Pages for server-side rendering and JavaScript for fast, responsive search
%\end{itemize}
%\textit{\textbf{Tools:} C\#, MongoDB, ASP.NET, Razor Pages, HTML/CSS/JS, Azure Build Pipelines}

%\divider

%\project{RIT COVID-19 Dashboard}{https://github.com/galenguyer/rit-covid-dashboard}
\noghproject{RIT COVID-19 Dashboard}{https://ritcoviddashboard.com}
\begin{itemize}
\item Wrote a Python scraper to check the official dashboard on an interval and update a SQLite database with the numbers from the official dashboard, then expose the saved data with a JSON API
\item Used the data from the scraper in a React app that provides all the information from the official dashboard as well as changes over a user-selected time period and graphs of all historical data
\item Used by roughly 10\% of the student body within 24 hours of launch
\end{itemize}
\textit{\textbf{Tools:} Python, SQLite, Flask, React, Docker}

%\project{PhotoSink}{https://github.com/galenguyer/PhotoSink}
%\begin{itemize}
%\item Re-wrote the lightweight photo gallery PhotoFloat, using modern technologies
%\item Used a python3 backend to construct the directory indexes and resize images into thumbnails for faster page loads
%\item Built a React app that requests the indexes and displays the gallery with thumbnails and folder names
%\item Used the url fragment for single page navigation so the full page doesn't need to be reloaded on navigation
%\end{itemize}
%\textit{\textbf{Tools:} Python, React, NGINX}

\divider

\noghproject{Global Content Delivery Network}{https://galenguyer.com}
\begin{itemize}
\item Consists of Azure Virtual Machines distributed around the globe to provide the least latency based on the user's geographical location
\item Deploys with Terraform for easily reproducible infrastructure
\item Uses Ansible to distribute static files, configuration files, and SSL certificates from a central server to all edge servers
\end{itemize}
\textit{\textbf{Tools:} Azure Virtual Machines, Azure Traffic Managers, Bash, Nginx}

\clearpage

\end{document}
