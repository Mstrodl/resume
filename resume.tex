%%%%%%%%%%%%%%%%%
% This is an sample CV template created using altacv.cls
% (v1.1.2, 1 February 2017) written by LianTze Lim (liantze@gmail.com). Now compiles with pdfLaTeX, XeLaTeX and LuaLaTeX.
%
%% It may be distributed and/or modified under the
%% conditions of the LaTeX Project Public License, either version 1.3
%% of this license or (at your option) any later version.
%% The latest version of this license is in
%%    http://www.latex-project.org/lppl.txt
%% and version 1.3 or later is part of all distributions of LaTeX
%% version 2003/12/01 or later.
%%%%%%%%%%%%%%%%

%% If you need to pass whatever options to xcolor
\PassOptionsToPackage{dvipsnames}{xcolor}

%% If you are using \orcid or academicons
%% icons, make sure you have the academicons
%% option here, and compile with XeLaTeX
%% or LuaLaTeX.
% \documentclass[10pt,a4paper,academicons]{altacv}

%% Use the "normalphoto" option if you want a normal photo instead of cropped to a circle
% \documentclass[10pt,a4paper,normalphoto]{altacv}

\documentclass[10pt,letter]{altacv}

%% AltaCV uses the fontawesome and academicon fonts
%% and packages.
%% See texdoc.net/pkg/fontawecome and http://texdoc.net/pkg/academicons for full list of symbols.
%%
%% Compile with LuaLaTeX for best results. If you
%% want to use XeLaTeX, you may need to install
%% Academicons.ttf in your operating system's font
%% folder.


% Change the page layout if you need to
\geometry{left=1cm,right=9cm,marginparwidth=7.25cm,marginparsep=0.75cm,top=0.5cm,bottom=1cm}

% Change the font if you want to.

% If using pdflatex:
\usepackage[utf8]{inputenc}
\usepackage[T1]{fontenc}
\usepackage[default]{lato}
\usepackage[none]{hyphenat}
\usepackage[document]{ragged2e}

% If using xelatex or lualatex:
% \setmainfont{Lato}

% Change the colours if you want to
\definecolor{AccentGreen}{HTML}{339966}
\definecolor{Purple}{HTML}{9429a0}
\definecolor{DarkerPurple}{HTML}{6e1f77}
\definecolor{SlateGrey}{HTML}{2E2E2E}
\definecolor{LightGrey}{HTML}{666666}
\definecolor{SomeBlue}{HTML}{4286f4}
\colorlet{heading}{Purple}
\colorlet{accent}{DarkerPurple}
\colorlet{emphasis}{SlateGrey}
\colorlet{body}{LightGrey}

% Change the bullets for itemize and rating marker
% for \cvskill if you want to
\renewcommand{\itemmarker}{{\small\textbullet}}
\renewcommand{\ratingmarker}{\faCircle}

\begin{document}
\name{Mary Strodl}
\tagline{Computer Scientist Seeking Full-Time in Summer 2025}
\personalinfo{%
  % Not all of these are required!
  % You can add your own with \printinfo{symbol}{detail}
  \email{mary@mstrodl.com}\\\smallskip
  \phone{(516) 654-5123}\\\smallskip
  \homepage{mstrodl.com}\\\smallskip
  \github{mstrodl}\\\smallskip
  %% You MUST add the academicons option to \documentclass, then compile with LuaLaTeX or XeLaTeX, if you want to use \orcid or other academicons commands.
%   \orcid{orcid.org/0000-0000-0000-0000}
}

%% Make the header extend all the way to the right, if you want.
\begin{fullwidth}
\marginpar{\makesidebarheader\medskip

\cvsection{Education}
\cvevent{Rochester Institute of Technology}{Software Engineering B.S.}{Expected Graduation May 2024}{}

\medskip

\cvsection{Skills}
\cvsubsection{Languages}
\cvtag{C\#}
\cvtag{Python}
\cvtag{HTML/CSS}
\cvtag{JavaScript}
\cvtag{Go}
\cvtag{React}
\cvtag{PowerShell}
\cvtag{Rust}
\cvtag{Java}
\cvtag{PHP}

\smallskip
\smallskip

\cvsubsection{Tools}
\cvtag{NGINX}
\cvtag{Linux}
\cvtag{Git}
\cvtag{GitHub}
\cvtag{Docker}
\cvtag{Docker Compose}
\cvtag{Traefik}
\cvtag{Azure}
\cvtag{Bash}
\cvtag{MongoDB}
\cvtag{Flask}
\cvtag{Ansible}
\cvtag{Terraform}
\cvtag{Kubernetes}
\cvtag{AWS Infinidash}

%% Yeah I didn't spend too much time making all the
%% spacing consistent... sorry. Use \smallskip, \medskip,
%% \bigskip, \vpsace etc to make ajustments.
\medskip
\cvsection{Activities}

\cvevent{Computer Science House}{Systems Administrator}{April 2020 - Present}{}
Maintain House Services such as the student-run server room and all services within it, including virtual machine and user profile management portals

\divider

\cvevent{WITR Radio}{Internal Developer}{October 2019 - Present}{}
Responsible for the upkeep and development of internal and public facing services such as the public website
%%\cvevent{Accessibility Learning Labs}{Software Engineer}{May 2020 - August 2020}{}
%%Worked in a federally funded research program to develop React apps to demonstrate the importance of accessible software to students and developers

\divider

\cvevent{Society of Software Engineers}{Public Relations}{January 2020 - May 2020, January 2021 - Current}{}
Coordinated company visits and tech talks throughout the semester

\divider

\cvevent{Boy Scouts}{Eagle Scout}{September 2017}{}
Achieved Boy Scouts' highest rank after earning 26 merit badges and designing, coordinating, and leading a 100+ hour service project

}
    \vspace*{-1\baselineskip}
\makecvheader
\end{fullwidth}
%% Provide the file name containing the sidebar contents as an optional parameter to \cvsection.
%% You can always just use \marginpar{...} if you do
%% not need to align the top of the contents to any
%% \cvsection title in the "main" bar.

%% Make the columns line up
\vspace{6pt}

\cvsection{Experience}

\cvevent{Bryx}{Software Engineer}{August 2022 -- Present}{Rochester, NY}
\begin{itemize}
  \item Maintaining a large React codebase with a Kotlin and Python backend
  \item Creating software to manage 5G modems in embedded Linux systems
  \item Building a new software update solution based on OSTree
\end{itemize}
\textit{\textbf{Tools:} React, Typescript, Python, Rust, Kotlin, C, Swift}

\divider

\cvevent{D3 Engineering}{Engineering Intern}{May 2022 -- August 2022}{Rochester, NY}
\begin{itemize}
  \item Developed camera drivers for Texas Instruments' RTOS Jacinto platform
  \item Maintained the Linux kernel and device tree for the Texas Instruments TDA4
  \item Improved custom build tooling, leading to a 50\% decrease in build times
\end{itemize}
\textit{\textbf{Tools:} Linux Kernel, C, RTOS, I2C}

\divider

\cvevent{Geisel Software}{Software Engineer Intern}{November 2021 -- May 2022}{Worcester, MA}
\begin{itemize}
  \item Collaborated with a team on a large PHP API and Vue web application
  \item Migrated internal projects from Vagrant to Docker containers
  \item Upgraded codebase to use database migrations
\end{itemize}
\textit{\textbf{Tools:} Nuxt, Vue, PHP, CodeIgniter 4}

\smallskip

\cvsection{Projects}

\smallskip

\project{Pixel Shader Programming Language}{mstrodl/anarchy}
\begin{itemize}
\item Designed a pixel shader programming language in Rust.
\item Built an editor with live preview, inline errors, and syntax highlighting using React and WebAssembly
\item Optimized interpreter to reach \char `~ 4ms per frame
\end{itemize}
\textit{\textbf{Tools:} WebAssembly, TypeScript, Rust}

\divider
  
\project{Gatekeeper Access Control System}{mstrodl/gatekeeper-core}
\begin{itemize}
\item Architected a door lock system to control access to Computer Science House special rooms using NFC (DESFire) tags to authenticate users
\item Leveraged Rust to ensure memory safety in a high-security environment
\item Wrote server software to pull data from LDAP and communicate with doors over MQTT
\end{itemize}
\textit{\textbf{Tools:} NodeJS, Rust, Embedded Systems, KiCad, MQTT, Express, MongoDB, Kubernetes}

\divider

\project{Flask}{computersciencehouse/devin}
\begin{itemize}
\item Published an Android app to buy drinks from CSH vending machines on the Google Play Store
\item Developed NFC support to allow unlocking doors using an Android phone
\item Created a robust OAuth2 client to authenticate and bill users
\end{itemize}
\textit{\textbf{Tools:} Kotlin, Jetpack Compose, NFC}

\clearpage

\end{document}
